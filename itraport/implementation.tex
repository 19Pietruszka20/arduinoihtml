Arduino Implementation:\newline
Initialize the Ethernet server library with port:\newline
EthernetServer server(80);\newline
How to set up IP address for my controller:\newline
IPAddress ip(192, 168, 0, 137);\newline
Start the Ethernet connection and the server:\newline
Ethernet.begin(mac, ip);\newline
Listen for incoming clients:\newline
EthernetClient client=server.available();\newline
\newline
\newline
We check connection with:\newline
if(client.connected())\newline
And then arduino sends request to a server with:\newline
client.print("GET /testcode/rower.php?v2=");\newline
client.print(v2);\newline
client.print("ANDprzebytadroga=");\newline %and&
client.print(przebytadroga);\newline
Standard http response header\newline
client.print(" ");\newline
client.print("HTTP/1.1");\newline
client.println();\newline
client.println("Connection: close");\newline
client.println();\newline
\newline
\newline
Html Implementation:\newline
Standard code need:\newline
<html></html>\newline
<body></body>\newline
<head></head>\newline
<!DOCTYPE HTML>\newline
<html lang="pl">\newline
<meta charset="utf-8"/>\newline
\newline
Css Implementation:\newline
We can add background by:\newline
body\newline
nawiasklamrowy\newline
	background-image:url('rower.jpg');\newline
	background-size:cover;\newline
	background-position:center;\newline
	position:relative;\newline
	overflow:hidden;\newline
nawiasklamrowy\newline
\newline
We should add Id by hashtag or class by dot. Then in id or class we should write width, height, color, text-allign, padding, background-color, font-size and other stuff. For example:\newline
hashtagwstep\newline
nawiasklamrowy\newline
	background-color:black;\newline
	color:black;\newline
	text-align:center;\newline
	padding: 15px;\newline
	letter-spacing: 2px;\newline
nawiasklamrowy\newline
\newline
The float command is usefull. The float CSS property places an element on the left or right side of its container, allowing text and inline elements to wrap around it. \newline
Also to clear float command we need to use clear:both.\newline
Creating charts:\newline
You need to create a php file that will connect to the database and retrieve data using:\newline
mysqli = new mysqli(DBHOST, DBUSERNAME, DBPASSWORD, DBNAME); with dollar at the beginning of course\newline
query = sprintf("SELECT v2, przebytadroga FROM rower"); with dollar at the beginning of course\newline
\newline
\newline
Also we need to create a javascript files. In it, the push() method adds more points to the array. Using the get method we download the php url file.
To create a chart, we need to create an instance of the Chart class with:\newline
<canvas id = "mycanvas" width = "400" height = "400"> </canvas>\newline
var ctx =  ("mycanvas");with dollar after equal symbol of course and hashtag before mycanvas\newline


We need to create an html file in which we need to include the appropriate files from javascript:\newline
<script type="text/javascript" src="js/jquery.min.js"></script>\newline
<script type="text/javascript" src="js/Chart.min.js"></script>\newline
<script type="text/javascript" src="js/app1.js"></script>\newline
\newline

The Internet is a worldwide collection of computer networks that began as a single network that was originally created in 1969 by ARPA (Advanced Research Projects Agency), a U.S. government agency that was far more interested in creating projects that would survive a nuclear war than in creating anything useful for the civilian population.

In its original form, ARPANET, the U.S. government hoped to create a network of computers that would allow communication between government agencies and certain educational centers that would be able to survive a nuclear explosion. It is doubtful that the original founders of ARPANET foresaw what we now know as "the Internet." From its humble beginnings as a military project, the ARPANET grew slowly throughout the 70's and 80's as a community of academics accomplished the truly monumental task of hammering out the building blocks of this new, open, modular conglomeration of networks.

In addition to the U.S. ARPANET, other countries also developed their own computer networks which quickly linked up to ARPANET, such as the UK's JANET (1983 onwards), and Australia's ACSnet (mid-1970s until replaced). Connecting these together would help develop a global internetwork.

The various protocols, including IP, TCP, DNS, POP, and SMTP, took shape over the years, and by the time the World Wide Web (HTML and HTTP) was created in the early 90's, this "Internet" had become a fully functional, fairly robust system of network communication, able to support this new pair of protocols which eventually turned the Internet into a household word.

While a large portion of users today confuse the Web with the Internet itself, it must be emphasized that the Web is only one type of Internet application, and one set of protocols among a great many which were in use for over a decade before the Web entered into the public awareness.
